% !TEX TS-program = pdfLaTeX
% !TEX encoding = UTF-8 unicode

\documentclass[11pt, a4paper, english]{article}

% Layout packages
\usepackage{changepage} %for margins
\usepackage{wrapfig} %for sidebars
\usepackage{tabularx} %for tables

\usepackage[noheadfoot,margin=1in]{geometry}
\setlength{\parindent}{0cm}
\setlength{\parskip}{2ex plus 0.5ex minus 0.2ex}


%Environment packages
\usepackage{graphicx} %for images
\usepackage{tcolorbox} %for background colours

%usability packages
\usepackage{todonotes} %for keeping track
\usepackage[l2tabu, orthodox]{nag} %forces LaTeX commands

%Environments

\newenvironment{wrapbox}[1][r]
	{\wrapfigure{#1}{0.35\textwidth}\tcolorbox}
	{\endtcolorbox\endwrapfigure}



\begin{document}

\title{Git Well Soon: a crash course in version control}

\maketitle

Ciaran has a set of SAR tools he is maintaining using a Github source code repository; [LINK] Here’s how you can get that code for your own use, make any changes you feel are needed and share it with anyone else who would require it.

\underline{What is Git?}

Git is a way of managing and keeping track of a piece of software’s history – who made what change when. It is also a way combining code from multiple developers together into a single piece of software, while still keeping track of who did what.

\underline{Installing Git}

If you're on Windows, then you can install Git for Windows via the University’s Program Installer. If you're on an Ubuntu or other Linux machine, then Git is usually pre-installed. If not, then you can install it with

\begin{verbatim}
$ sudo apt-get install git-all
\end{verbatim}

on the command line.

\begin{wrapbox}
From here out, it’s assuming that you are using a standard Bash terminal for navigation. If you are on Windows, a good enough emulation comes with Git for Windows; just type ‘Git Bash’ into the search bar.
\end{wrapbox}

\begin{wrapbox}
If you’re not familiar with using a terminal to navigate, all you need for this session are the following commands: \verb|$cd [folder]| will move you into [folder]; \verb|$cd ..| will move you up to the parent folder; \verb|$ls| will show files in the current folder; and \verb|mkdir [name]| will create a new directory called [name].
\end{wrapbox}

\underline{Getting the code}

First, navigate to a convenient directory using cd and mkdir, as appropriate. Once there:

\begin{verbatim}

$git clone [LINK]

\end{verbatim}

This is now the \textbf{root} of your local repository – as far as Git is concerned, everything is relative to this point.

\missingfigure{diagram of clone}

\underline{What is happening here}

This will copy all of Ciaran’s code from Github to your machine into a \textbf{local repository}, complete with the history and branches [see later] of his project right to the beginning – you can view this with the command \verb|$ git log|. Press ‘q’ to exit the log viewer.

\underline{Why do this?}

As well as getting the entire project’s codebase, when you cloned the repository with \verb|$ git clone|, you also cloned its history and configuration. This means that you can:

\begin{itemize}
\item Keep it up to date with any revisions Ciaran might make in the future
\item Review any changes he might have made in the past, and understand why he made those changes.
\item If you make a huge, unfixable mistake, or if the code was working last Tuesday and isn’t now, you can roll the entire project back to a point where everything was working
\item Fix any bugs or glitches you might find
\item Add any further features that you think the code might need, and share those features with anyone else.
\end{itemize}


\end{document}

